\documentclass[a4paper,10pt]{article}
\usepackage[utf8]{inputenc}
\usepackage[cm]{fullpage}
\usepackage{amsmath}
\usepackage{amssymb}

%opening
\title{Assembly 2 Docs }
\author{Hamish}

\begin{document}

\maketitle

\begin{abstract}

Documents for personal use, for keeping tracks of the maths behind the Assembly 2 work bench for FreeCAD~v0.15.

\end{abstract}

\tableofcontents

% \vfil

% \pagebreak


\section{Solver approach}

Adjust the placement variables (position and rotation variables, 6 for each object) as satisfy all constraints;
by adding one constraint at a time and adjusting the degrees-of-freedom.
Ideally, this degrees-of-freedom can be identified, so that they can be adjusted with having to check previous constraints.
For non-simple systems this is not practical however as there are to many combinations to hard-code.

Therefore a hierarchical constraint system is used.
When attempting the solve the current constraint, the placement variables are also adjusted/refreshed according to the previous constraints to allow for non-perfect degrees-of-freedom.
This is done in a hierarchical way, with parent constraints minimally adjusting the placement variables as to satisfy the assembly constraints.

\section{Plane mating constraint}

2 parts 
\begin{enumerate}
 \item rotating objects as to align selected faces (not done if face and vertex are selected), \textbf{axis alignment union}
 \item moving the objects as to specified offset is satisfied, \textbf{plane offset union}
\end{enumerate}


\section{plane offset union - analytical solution}

inputs
\begin{description}
 \item[~~~~$\mathbf{a}$] - normal vector of reference face.
 \item[~~~~$\mathbf{p}$] - point on object 1.
 \item[~~~~$\mathbf{q}$] - point on object 2. 
 \item[~~~~$\alpha$] - specified offset
 \item[~~~~$\mathbf{d}_1,\mathbf{d}_2...$] - linear motion degree-of-freedom for object 1/2 (max of 3, min of 1)
\end{description}
required displacement in the direction of  $\mathbf{a}$:
\begin{equation}
 r = \mathbf{a} \cdot (\mathbf{p} - \mathbf{q}) - \alpha
\end{equation}
require components for each $\mathbf{d}$, $v_1,v_2,...$ therefore equal to
\begin{equation}
 \mathbf{a} \cdot ( v_1 \mathbf{d}_1 +  v_2 \mathbf{d}_2 +  v_3 \mathbf{d}_3 )  = r
\end{equation}
which has infinite solutions if more than one degree-of-freedom with $\mathbf{d} \cdot \mathbf{a} \ne 0 $

Therefore looking for least norm solution of:
\begin{equation}
 r = a_x ( v_1 d_{1,x} + v_2 d_{2,x } + \dots) + a_y ( v_1 d_{1,y} + v_2 d_{2,y } + \dots) + a_z ( v_1 d_{1,z} + v_2 d_{2,z } + \dots)
\end{equation}
which gives
\begin{align}
 \begin{bmatrix}
  (a_x d_{1,x} + a_y d_{1,y} + a_z d_{1,z}) &
  (a_x d_{2,x} + a_y d_{2,y} + a_z d_{2,z}) &
  (a_x d_{3,x} + a_y d_{3,y} + a_z d_{3,z})  
 \end{bmatrix}
 \begin{bmatrix}
  v_1  \\ v_2 \\ v_3 
 \end{bmatrix} 
 = [ r ] \\
 \begin{bmatrix}
  \mathbf{a} \cdot \mathbf{d}_1 & \mathbf{a} \cdot \mathbf{d}_2 & \mathbf{a} \cdot \mathbf{d}_3 
 \end{bmatrix}
 \begin{bmatrix}
  v_1  \\ v_2 \\ v_3 
  \end{bmatrix}  = r \\ 
 A \mathbf{v} = [r]
\end{align}
then solve for least norm using numpy.linalg.lstsq






\end{document}
